\documentclass[11pt ,twoside, a4paper]{article}

\begin{document}

\begin{itemize}
	\item Hazus Flood Model
	\begin{itemize}
		\item computing damage of flood 
	\end{itemize}
	
	\item loss estimation process
	\begin{itemize}
		\item maps, tables, graphs, \& reports
		\item economic impacts by intersecting hazard and inventory
		\item inventory at risk
		\item flood hazard
	\end{itemize}
	
	\item agriculture
	\begin{itemize}
		\item data on crop times and quantities
		\item top 20 crops for each NRI regions
		\item does not include livestock
		\item Hazus-MH removes urban areas, wetlands, forests, etc.
	\end{itemize}
	
	\item vehicles
	\begin{itemize}
		\item loss is based on deaths
	\end{itemize}
	
	\item DEM-digital election model
	\begin{itemize}
		\item DEM files are simple regularly spaced grid of elevation points
		\item grid spacing, coordinates, etc.
		\item commonly builds using remote sensing techniques \& land surveying 
	\end{itemize}
	
	\item National Elevation Dataset (NED)
	\begin{itemize}
		\item dataset produced by USGS 
		\begin{itemize}
			\item 1 Arc-second resolution for most of US
		\end{itemize}
	\end{itemize}
	
	\item DEM guidlines
	\begin{itemize}
		\item Cell size- higher resolution input -> higher resolution output, but takes longer to process
	\end{itemize}
	
	\item watersheds
	\begin{itemize}
		\item area that catches water
	\end{itemize}
	
	\item Riverine req
	\begin{itemize}
		\item issue: need to ensure that there's enough train to perform hydraulic analysis
	\end{itemize}
	
	\item Coastal req
	\begin{itemize}
		\item issue: need to ensure that there's enough train to perform transect analysis
		\item convex hull of costal region
	\end{itemize}
	
	\item stream delineation
	\begin{itemize}
		\item build a synthetic stream network 
	\end{itemize}
	
	\item Error handling of stream delineation
	\begin{itemize}
		\item DEM terrain can predict incorrectly
		\item to combat this, artificial elevation is placed to match the observed train
	\end{itemize}
	
	\item hydrologic analysis
	\begin{itemize}
		\item objective: determine discharge values in streams
	\end{itemize}

	\item gages
	\begin{itemize}
		\item determine water level and discharge of water
	\end{itemize}
	
	\item regression
	\begin{itemize}
		\item inputs:
		\item topographic
		\begin{itemize}
			\item drainage area
			\item mean basin elevation + slope
			\item basin length
			\item channel length
		\end{itemize}
		\item other parameters
		\begin{itemize}
			\item temperature
			\item precipitation
			\item soil type
			\item forest cover
			\item snowfall
		\end{itemize}
	\end{itemize}
	
	\item Depth-Damage Functions
	\begin{itemize}
		\item determine the amount of damage that could result given the type of structure, etc. that results with depth of flood
	\end{itemize}
	
	\item loss estimation 
	\begin{itemize}
		\item loss is weighted by the elevation of the building during the flood
	\end{itemize}
	
	\item Hazus flood casualties
	\begin{itemize}
		\item Hazus flood model does not generate casualty estimates
		\item Hazus Tsunami model does generates casualty estimates
	\end{itemize}
	
	\item Average Annualized Loss (AAL)
	\begin{itemize}
		\item estimated direct economic losses averaged on an annual basis using future direct economic losses
		\item used to determine the financial feasibility o mitigation projects
	\end{itemize}
	
	\item storm surge
	\begin{itemize}
		\item flooding as a result of a storm (mostly hurricanes)
	\end{itemize}

	\item 
	\begin{itemize}
		\item 
	\end{itemize}


\end{itemize}

\end{document}







